\documentclass[]{book}
\usepackage[none]{hyphenat}
\usepackage{amsfonts}
\usepackage{mathtools}
\usepackage{amsthm}
\usepackage{amssymb}
\usepackage{amsmath}
\usepackage[english]{babel}
\usepackage{imakeidx}
\usepackage{afterpage}
\usepackage{enumerate}
\usepackage{emptypage}
\usepackage{calrsfs}
\usepackage{stmaryrd}
\usepackage{mathpazo}
\usepackage{pst-node}
\usepackage{tikz-cd}

\DeclareMathAlphabet{\pazocal}{OMS}{zplm}{m}{n}
%\pagenumbering{gobble}


%No section numberint %
%\renewcommand{\thesection}{\arabic{section}}


%------  MARGINS 


\usepackage{geometry}
\geometry{
	a4paper,
	total={170mm,257mm},
	left=20mm,
	top=20mm,
}

%-------------INDENTATION
\setlength\parindent{0pt}

%----------------------------------ENVIRONMENTS ----------------------------%

\theoremstyle{definition}
\newtheorem{defin}{Definition}[section]


\theoremstyle{definition}
\newtheorem{prb}{Problem}

\theoremstyle{definition}
\newtheorem{thm}{Theorem}[section]

\theoremstyle{definition}
\newtheorem*{remark}{Remark}

\theoremstyle{definition}
\newtheorem{cor}{Corollary}[section]

\theoremstyle{definition}
\newtheorem{lem}{Lemma}

\theoremstyle{definition}
\newtheorem*{obs}{Obserbation}

\theoremstyle{definition}
\newtheorem{prop}{Proposition}

\theoremstyle{definition}
\newtheorem{ex}{Example}

\theoremstyle{definition}
\newtheorem{exercise}{Excercise}

\theoremstyle{definition}
\newtheorem*{construction}{Construction}

\newcommand\blankpage{%
	\null
	\thispagestyle{empty}%
	\addtocounter{page}{-1}%
	\newpage}


%----------------------------FONT-----------------------------%

\usepackage{fontspec}
\defaultfontfeatures{Mapping=tex-text,Scale=MatchLowercase}
\setmainfont{TeX Gyre Pagella}

%---------------------------- COMMANDS -----------------------------%

\newcommand{\ita}[1]{\textit{#1}}
\newcommand{\Q}{\mathbb{Q}}
\newcommand{\R}{\mathbb{R}}
\newcommand{\Z}{\mathbb{Z}}
\newcommand{\C}{\mathbb{C}}
\newcommand{\N}{\mathbb{N}}
\newcommand{\proj}{\mathbb{P}}
\newcommand{\cat}{\mathcal{C}}
\newcommand{\caat}{\mathcal{D}}
\newcommand{\catone}{\mathcal{C}_1}
\newcommand{\cattwo}{\mathcal{C}_2}
\DeclareMathOperator{\obj}{Obj}
\DeclareMathOperator{\homc}{Hom_{\mathcal{C}}}
\DeclareMathOperator{\homprime}{Hom_{\mathcal{C'}}}
\DeclareMathOperator{\homd}{Hom_{\mathcal{D}}}
\DeclareMathOperator{\homone}{Hom_{\mathcal{C}_1}}
\DeclareMathOperator{\homtwo}{Hom_{\mathcal{C}_2}}
\newcommand{\bigslant}[2]{{\raisebox{.2em}{$#1$}\left/\raisebox{-.2em}{$#2$}\right.}}
\newcommand*\quot[2]{{^{\textstyle #1}\big/_{\textstyle #2}}}
%---------------------- STRUCTURE -----------------------

%\chapter{}
%\section{}


%----------------------- DOCUMENT---------------------------

\title{%
	Geometry and Topology of Manifolds \\
	\large Master in Advanced Mathematics}
\date{2019-2020}
\author{Alejandro Abregú}

\begin{document}
	
%\afterpage{\blankpage}
	
\maketitle
\thispagestyle{empty}
\tableofcontents
\thispagestyle{empty}
\clearpage
\pagenumbering{arabic} 
\chapter{Manifolds}
\section{Introduction}
\begin{defin}
  An $n$-dimensional topological manifold is a topological space $M$ satisfying
  \begin{enumerate}[i)]
    \item $M$ is second countable, i.e. a topological space whose topology has a countable base.
    \item $M$ is Hausdorff
    \item $M$ is locally homeomorphic to $\R^{n}$ (any point of $M$ has a neighborhood which is
      homeomorphic to $\R^{n}$ )
  \end{enumerate}
\end{defin}
\begin{ex}
  \begin{enumerate}[i)]
    \item $\R^{n}$
    \item $S^{n}$
    \item If $X,Y$ are topological manifolds, then $X \times Y$ is an $(m \times  n)$ topological
      manifold.
  \end{enumerate}
\end{ex}
\begin{defin}
  A chart on a topological $n$-dimensional manifold $M$ is a pair $(U,\varphi)$ where $U \subseteq M$
  is an open and $\varphi:U\longrightarrow \R^{n} $ is a continuous map inducing a homeomorphism
  \[
    U\longrightarrow \varphi(U) 
  \] 
\end{defin}
\begin{remark}
  \begin{enumerate}[i)]
    \item If $(U,\varphi)$ is a chart, then $\varphi(U)\subseteq \R^{n}$ is open
    \item If $X,Y$ are topological manifolds of dimension $m,n$ respectively, and $X \cong Y$, then
      "m=n"
  \end{enumerate}
\end{remark}
\begin{defin}
  An atlas on a topological manifold $M$ is a collection of chats $\lbrace (U_i,\varphi_i
  \rbrace_{i\in I} $ on $M$ satisfying:
  \begin{enumerate}[i)]
    \item $M=\bigcup_{i\in I}^{} U_{i}$
    \item For any $i,j\in I$ the map
      \[
        \varphi_j\circ \varphi_i^{-1}:\varphi(U_i \cap U_j) \longrightarrow  \varphi(U_i,U_j)
      \] 
      is $C^{\infty}$ (we call it smooth).
  \end{enumerate}
\end{defin}

\begin{center}
 \fbox{\begin{minipage}{10em}
     \begin{center}
      FIGURE 1
     \end{center}
\end{minipage}}
\end{center}
A chart $(U,\varphi)$ is compatible with an atlas $\mathcal{A}$ if and only if $\mathcal{A}\cup
\lbrace (U,\varphi) \rbrace $ is again an atlas.

An atlas is maximal if it contains all charts compatible with it. It $\mathcal{A}$ is an atlas,
then
\[
\mathcal{A}\cup \lbrace (U,\varphi);(U,\varphi) \text{ is a chart compatible with $\mathcal{A}$} \rbrace 
\] 
is a maximal atlas. A smooth structure on a topological manifold is a maximal atlas on it.

\begin{defin}
  A smooth manifold is a pair $(M,\mathcal{A})$ where $M$ is a topological manifold, and
  $\mathcal{A}$ is a smooth structure (that is, maximal atlas) on $M$
\end{defin}

\begin{obs}
  Usually we will omit $\mathcal{A}$ from the notation and write $M$ instead of $(M,\mathcal{A})$			
\end{obs}

Since any atlas on a topological manifold is contained in a unique max atlas to specify a smooth
structure on a topological manifolds if suffices to specify an atlas

\begin{defin}
  A homomorphism from a smooth manifold $(M,\mathcal{A})$ to another manifold $(N,\mathcal{B})$ is
  a homomorphism $f:M\longrightarrow N $ such that for any chart $(U,\varphi)$ of $\mathcal{A}$,
  $(f(U),\varphi\circ f^{-1})$ is a chart of $\mathcal{B}$. Then $f^{-1}$ is also a diffoemorphism
\end{defin}

\begin{defin}
  Two smooth manifolds $M,N$ are diffeomorphic if and only iff there exists a diffeomorphism from
  one to another.
\end{defin}

\begin{enumerate}[i)]
  \item Does always a topological manidold admit a smooth structure? (Yes for dim<4, no otherwise)
  \item Can a topological manifold admit two smooth structures which are not diffeomorphic? (No for
    dim<4, yes otherwise)
  \item Does  $\R^{n}$ admit two different smooth structures? (No except for $n=4$ )
\end{enumerate}

\begin{ex}
  \begin{enumerate}[i)]
    \item If $M=\R^{n}$, then $\mathcal{A}=\lbrace (\R^{n},id) \rbrace $ is an atlas on $M$ which is
      contained, of course, in a unique maximal atlas $\mathcal{A'}$. Let $f: \R^{n} \longrightarrow
      \R^{m}$ is a homeo which is not diffeomorphism
    (for example $f(x_1,...,x_{n})=(x_1^3,....,x_{n}^3)$). Then $\mathcal{B}=\lbrace (\R^{n},f)
    \rbrace $ is another atlas on $M$, contained in a maximal atlas $\mathcal{B'}$. 
    We have $\mathcal{A}\neq\mathcal{B'}$ because $\lbrace (\R^{n},id),(\R^{n},f) \rbrace $ is not
    an atlas. However $(\R^{n},\mathcal{A}),(\R^{n},\mathcal{B})$ are diffeomorphic.

  \item $M=S^{n}$. Observe that $S^{n}$ is not homeo (open or not) to any subset of $\R^{n}$, so any
      atlas on $S^{n}$ has to have at least 2 charts. Given $x\in S^{n}$, let
      $U_{x}=S^{n}\setminus \lbrace x \rbrace $ and define 
      \[
      x^{\perp}=\lbrace y\in \R^{n+1}, <x,y>=0 \rbrace \cong \R^{n}
      \] 
      and let
      \begin{align*}
        \varphi: U_x &\longrightarrow x^{\perp}\\
        z &\longmapsto x^{\perp} \cap \lbrace \lambda z + (1-\lambda);\lambda\in\R \rbrace 
      \end{align*}
      \begin{center}
       \fbox{\begin{minipage}{10em}
           \begin{center}
            Figure 2
           \end{center}
      \end{minipage}}
      \end{center} 
  \end{enumerate}
\end{ex}
\begin{exercise}
  \begin{enumerate}[i)]
    \item Prove that $(U_x,\varphi_x)$ is a chart on $S^{n}$ for all $x\in S^{n}$.
    \item If $x\neq x'$ are points of $S^{n}$, then $A=\lbrace (U_x,\varphi_x),(U_{x'},\varphi_{x'})
      \rbrace $ is an atlas on $S^{n}$.
      \item $M=\R P^{n} = \lbrace L\subseteq \R^{n+1}\mid L \text{ 1-dim subspace of } \R^{n+1}
        \rbrace$. Given $L,L'\in \R P^n$, define
        \[
          d(L,L')=\inf_{\substack{ x\in L\cap S^n\
             }} \lbrace || x-x' ||   \rbrace 
        \]
        Then $d:\R P^n \times \R P^n\longrightarrow \R $ is a distance. Hence, we get a topology
        on $\R P^n$.
    \end{enumerate}
\end{exercise}
Let's define an atlas on $\R P^n$ (which is a $n$-dimensional topological manifold).
\begin{defin}
  Let $L\in \R P^n$ and let $H\subseteq \R^{n+1}$ a codimension 1 linear subspace such that
  $L\cap H=\lbrace 0 \rbrace $ (Hence $\R^{n+1}=L \oplus H$). Let
  \[
  U_{L,H}= \lbrace \Lambda  \in \R P^n \mid \Lambda \cap H = \lbrace 0 \rbrace  \rbrace 
  \]
  and let
 \begin{align*}
 \varphi_{L,H}: U_{L,H} &\longrightarrow \text{Hom}_{\R}(H,L)\cong \R^{n} \\
 \Lambda &\longmapsto \psi:L\longrightarrow H  
 \end{align*}
 defined by the condition that $\Lambda=\lbrace (l,\psi(l)) \mid l\in L \rbrace \subset L \oplus H=
 \R^{n+1}$
\end{defin}
\begin{exercise}
  Prove that
  \begin{enumerate}[i)]
    \item $U_{L,H}\subseteq \R P^n$ is open
    \item $(U_{L,H},\varphi_{L,H})$ is a chart
    \item $\lbrace (U_{L,h)},\varphi_{L,H} \rbrace_{L,H}$ is an atlas on $\R P^n$ 
  \end{enumerate}
\end{exercise}
\begin{ex}
   \[
     Gr(d,n)= \lbrace V \subset \R^{n} \mid V \text{ d-dim $\ell$ linear subspace} \rbrace \quad  (\R
     P^n=Gr(1,n+1)
  \]
  \[
    V,V'\in Gr(r,n),\quad d(V,V')= \sup_{\substack{ x\in V\cap S^{n+1}\
         }} \lbrace  \inf_{\substack{ x'\in V'\cap S^{n-1}\
              }} \lbrace || x-x'||   \rbrace 
            \rbrace 
   \] 
   this defines a distance and hence a topology on $Gr(r,n)$. Then $G(r,n)$ is a topological
   manifold of dimension $r(n-r)$
\end{ex}

\begin{exercise}
  Generalize the previous constructions and construc an atlas on $Gr(r,n)$
\end{exercise}

\begin{construction}
   \begin{enumerate}[i)]
     \item If $(M,\mathcal{A},(U,\mathcal{B}$ are smooth manifolds, then
       \[
         \mathcal{A}\times \mathcal{B}= \lbrace (U \times  V, \varphi \times \psi) \mid
         (U,\varphi)\in \mathcal{A}, (V,\psi)\in \mathcal{B} \rbrace
       \] 
     \item Connected sum.

       Let $M,N$ be smooth connected manifolds of the same dimension $n$. Let $p\in M$ and $q\in N$ be
       any points. Let $(U,\varphi),(V,\psi)$ be charts of $M,N$ respectively and such that $p\in
       U,q\in V$. We may assume $\varphi(p)=0=\psi(q)$.
      
       Let $\epsilon >0 $ be small enough so that
       \[
         B_{\epsilon} \subset \varphi(U)\cap \psi(V)
       \] 
       Let $M_0=M\setminus \varphi^{-1}(\overline{B_{\epsilon^2}})$, $N_0=N\setminus
       \psi^{-1}(\overline{B_{\epsilon^2}})$

       Let $\sim$ be the equivalence relation on $M_0\coprod N_0$ generated by the pairs $(a,b)$ 
       such that $a\in M_0, b\in N_0$ 
       \begin{enumerate}[i)]
         \item $a\in \varphi^{-1}, b\in\psi^{-1}(B_{\epsilon})$
         \item $\frac{\varphi(a)}{|| \varphi(a) || }= \frac{\psi(b)}{|| \psi(b) || }$ 
         \item $|| \varphi(a) || \cdot || \psi(b) || = \epsilon^3 \quad \quad R \subset (M_0 \coprod
           N_0)\times (M_0 \coprod N_0)$
       \end{enumerate}
       Define $M\#N= \quot{M_0\coprod N_0}{\sim}$ 
       We thus get a topological space.

       We take on $M\#N$ the smooth structure induced by the atlas
       \[
         \lbrace (U,\varphi) \mid \text{either $U \subset M_0$ and $(U,\varphi)$ is a chart of $M$ or
         $U\subset N$ and $(U,\varphi)$ is a chart of $N$}\rbrace 
       \] 
   \end{enumerate}
  \end{construction}
   \begin{thm}
      $M\# N$ is a well defined smooth $n$-manifold.
   \end{thm}
   \begin{thm}
     \begin{enumerate}[i)]
       \item Any compact and connected $1$-dimensional manifold is diffeomorphic to $S^{1}$ 
       \item Any compact connected $2$-dim manifold is diffeomorphic to
         \[
             \left\{
                \begin{array}{ll}
                  S^{2}\\
                  (S^{1}\times S^{1})\#...\#(S^{1}\times S^{1}) \\
                  \R P^2  \# (S^{1}\times S^{1})\#...\#(S^{1}\times S^{1})
                \end{array}
             \right.
         \] 
     \end{enumerate}
   \end{thm}
   \section{Smooth Functions}
   $M$ is a smooth manifold of $\text{dim} \geq 1$ and $\mathcal{A}=\lbrace (U_{i},\phi_i \rbrace
   _{i\in I}$ an atlas
   \begin{itemize}
     \item $U$ is an open of $M$.
     \item $\phi_i:U_i\longrightarrow \R^{n}$ is a homoemorphism on the image $U_i \cong \phi_i(U_i)$ 
     \item Compatible change of charts: $\forall i,j \in I, \phi_j\circ \phi_i^{-1}$ is $C^{\infty}$
   \end{itemize}
   \begin{center}
    \fbox{\begin{minipage}{10em}
        \begin{center}
         Figure 3
        \end{center}
   \end{minipage}}
   \end{center}
   
   smooth function $f:M\longrightarrow \R $ 
   \begin{center}
    \fbox{\begin{minipage}{10em}
        \begin{center}
          Figure 4
        \end{center}
   \end{minipage}}
   \end{center}

   Define $f_{i}=f\circ \phi_{i}^{-1}$. Since $f_{i}\circ \phi_{i}=f$, we have that $f_i= f\circ
   \phi_{j}^{-1}=f_{i}\circ \phi_{i}\circ \phi_{j}^{-1}$.
   The following are equivalent:
   \begin{enumerate}[i)]
     \item There exist  $i\in I$ such that $f_{i}$ is $C^{\infty}$ 
     \item $\forall  i\in I,\quad f_{i}$ is $C^{\infty}$
   \end{enumerate}

   \begin{defin}
     $f:M\longrightarrow \R$ is smooth at $p\in M$ if $(i)$ or $(ii)$ is satisfied.

     The collection of all smooth functions will be denoted by $F(M)$ or $C^{\infty}(M)$. $F(M)$ is
     a $\R$-vector space
     \begin{itemize}
       \item $(f+g)(p)=f(p)+g(p)$ 
       \item $(\lambda f)(p) =\lambda f(p)$
     \end{itemize}
     $\forall f,g\in F(M), \forall \lambda \in R$
   \end{defin}
   \section{Partitions of unity}

  \begin{defin}
   If $f:M\longrightarrow \R $ is a smooth function, then we define 
   \[
     \text{supp(f)}=\lbrace p\in M \mid f(p)\neq 0 \rbrace 
   \]     
  \end{defin}

  \begin{defin}
    A partition of unity on a smooth manifold is a collection $\lbrace \varphi_i \rbrace _{i\in I}$
    of smooth functions on $M$ such that
 
  \begin{enumerate}[i)]
    \item $0 \leq  \varphi_{i}(p)\leq 1,\quad \forall  i\in I ,\quad \forall p\in M$
    \item Every $p\in M$ has an open neighborhood $V$ such that $V \cap
      \text{supp}(\varphi_{i})=\emptyset,\quad \forall  i\in I$ except for a finite number of $i$
    \item $\sum_{i\in I}^{} \varphi_{i}(p)=1,\quad \forall  p\in M $
  \end{enumerate}
  A partition of unity is subordinated to an open cover of $M$ if $\forall  i\in I \exists u\in U$
  such that $\text{supp}(\varphi_{i})\subset U$
  \end{defin}
\end{document}
